\documentclass[10pt,a4paper]{article}

% ------------------------- PREAMBUŁA -------------------   % stała ścieżka względna do katalogu z  obrazkami.
\usepackage{graphicx}
%\input{settings/packages}  
%\graphicspath{{images/}} 
\usepackage{graphicx}
\usepackage{subcaption}
\usepackage[polish]{babel}
\usepackage[T1]{fontenc}
\usepackage[utf8]{inputenc}
\usepackage{amsmath, amsfonts, amssymb}
\usepackage{booktabs}
\usepackage[top=2.5cm, bottom=2.5cm, left=2cm, right=2cm]{geometry}
\usepackage{textcomp}
\usepackage{gensymb}
\usepackage{textgreek}
\usepackage{geometry}
\usepackage{pdflscape}
\usepackage{pdfpages}
\usepackage{hyperref}
\usepackage{xcolor}

%--- METADANE
%\begin{figure}
%	\centering
%	\includegraphics[width=0.7\textwidth]{gik.png}
%\newcommand{\logoGIK}{photos\gik.png}
%\end{figure}

\title{Transformacje}
\author{Adam Buława, numer indeksu 325727 \\ Jakub Fajfer numer indeksu 325742}

\begin{document}
	\maketitle{Transformacje} %utworzy stronę tytułową
\tableofcontents
\newpage	
\section{Wprowadzenie}

\subsection{Cel ćwiczenia}
Cel ćwiczenia to utworzenie programu implementującego podane transformacje geodezyjne:
\begin{itemize}
	\item XYZ (geocentryczne) -> BLH
	\item  BLH -> XYZ 
	\item  XYZ -> NEUp
	\item  BL(GRS80, WGS84, ew. Krasowski) -> 2000 
	\item  BL(GRS80, WGS84, ew. Krasowski) -> 1992
\end{itemize}
przy wykorzystaniu zdalnego, publicznego repozytorium obsługiwalnego przez dwie osoby. Program miał zawierać odpowiednie klauzule, usprawniające i ułatwiające działanie programu, umożliwiać podawanie argumentów przy wywołaniu, oraz wczytywać dane z pliku wejściowego. Efektem końcowym działania programu powinien być plik wynikowy zawierający żądane dane.
\subsection{Wykorzystane programy}
Do pracy nad skrytem, opisem przebiegu ćwiczenia i instrukcją do aplikacji zostały użyte programy:
\begin{itemize}
	\item Środowisko programistyczne Spyder, Python 3.11
	\item TeXstudio jako edytor ułatwiający tworzenie dokumentów
	\item GitHub, jako serwis internetowy do tworzenia zdalnego repozytorium
\end{itemize}
\section{Etapy pracy}
\subsection{Utworzenie zdalnego repozytorium}
Pracę rozpoczęliśmy od utworzenia zdalnego repozytorium, które umożliwia edycję plików o różnych rozszerzeniach z pozycji różnych urządzeń. Repozytorium powstało w aplikacji GitHub.
\subsection{Kod programu}
OPIS CO JEST GRANE PO KOLEI, 
\begin{itemize}
	\item po co i co to TRANSFORMATIONS plus działanie bibliotek, instalacja itp
	\item użycie funkcji i algorytmów z gw di transformacji, przykład (nazwafunkcji(argumenty)), jakie biblioteki i użyte bo bardzo ułatwiają pracę oraz estetykę 
	\item {if name == main} co to i po co, przekazywanie arugmentów przez wiersz poleceń, wbijanie pliku do wiersza poleceń, nazwanie pliku wyjściowego
	\item ochrona przed błędami w stylu to nie działa bo cośtam, TRZEBA ERRORY DOROBIĆ zastosowanie klauzuli 'try' - 'except (...)Error' z oddaniem komunikatu w przypadku 'except'. 
	\item próby i wklepywanie plików tekstowych, zgodnie z naszym życzeniem:)
\end{itemize}
\subsection{Dokumentacja ćwiczenia}
Instrukcja do programu została zapisana w pliku Read.me w zdalnym repozytorium.
\section{Podsumowanie}
\subsection{Umiejętności nabyte podczas pracy nad projektem}
\subsection{Trudności i obserwacje}
\subsection{Wnioski}
\subsection{Link do repozytorium}


\end{document}