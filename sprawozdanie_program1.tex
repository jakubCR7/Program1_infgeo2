\documentclass[10pt,a4paper]{article}

% ------------------------- PREAMBUŁA -------------------   % stała ścieżka względna do katalogu z  obrazkami.
\usepackage{graphicx}
%\input{settings/packages}  
%\graphicspath{{images/}} 
\usepackage{graphicx}
\usepackage{subcaption}
\usepackage[polish]{babel}
\usepackage[T1]{fontenc}
\usepackage[utf8]{inputenc}
\usepackage{amsmath, amsfonts, amssymb}
\usepackage{booktabs}
\usepackage[top=2.5cm, bottom=2.5cm, left=2cm, right=2cm]{geometry}
\usepackage{textcomp}
\usepackage{gensymb}
\usepackage{textgreek}
\usepackage{geometry}
\usepackage{pdflscape}
\usepackage{pdfpages}
\usepackage{hyperref}
\usepackage{xcolor}

\title{Transformacje}
\author{Adam Buława, numer indeksu 325727 \\ Jakub Fajfer, numer indeksu 325742}

\begin{document}
	\maketitle{Transformacje}
\tableofcontents
\newpage	
\section{Wprowadzenie}

\subsection{Cel ćwiczenia}
Cel ćwiczenia to utworzenie programu implementującego podane transformacje geodezyjne:
\begin{itemize}
	\item XYZ (geocentryczne) -> BLH
	\item  BLH -> XYZ 
	\item  XYZ -> NEUp
	\item  BL(GRS80, WGS84, ew. Krasowski) -> 2000 
	\item  BL(GRS80, WGS84, ew. Krasowski) -> 1992
\end{itemize}
przy wykorzystaniu zdalnego, publicznego repozytorium obsługiwalnego przez dwie osoby. Program miał zawierać odpowiednie klauzule, usprawniające i ułatwiające działanie programu, umożliwiać podawanie argumentów przy wywołaniu, oraz wczytywać dane z pliku wejściowego. Efektem końcowym działania programu powinien być plik wynikowy zawierający żądane dane.
\subsection{Wykorzystane programy}
Do pracy nad skrytem, opisem przebiegu ćwiczenia i instrukcją do aplikacji zostały użyte programy:
\begin{itemize}
	\item Środowisko programistyczne Spyder, Python 3.11
	\item TeXstudio jako edytor ułatwiający tworzenie dokumentów
	\item GitHub, jako serwis internetowy do tworzenia zdalnego repozytorium
\end{itemize}
\section{Etapy pracy}
\subsection{Utworzenie zdalnego repozytorium}
Pracę rozpoczęliśmy od utworzenia zdalnego repozytorium, które umożliwia edycję plików o różnych rozszerzeniach z pozycji różnych urządzeń. Repozytorium powstało w aplikacji GitHub.
\subsection{Kod programu}
\begin{itemize}
	\item Do poprawnego działania programu niezbędne jest zainstalowanie w konsoli modułu Python oraz biblioteki Numpy, z której skorzystaliśmy przy tworzeniu programu. Biblioteka ta umożliwia działania na tablicach, które znacznie upraszczają obliczenia. Początek naszego kodu stanowi zdefiniowanie klasy Tranformations zawierającej modele elipsoid. Użytkownik programu może wybrać jeden z modeli, do których przypisane są odpowiednie wartości parametrów elipsoid.
	\item W dalszej części naszego kodu zapisaliśmy definicje funkcji wykonujących poszczególne transformacje. Każda z funkcji przyjmuje argumenty w zależności od wybranego przez użytkownika modelu elipsoidy. \newline
	hirvonen - funkcja zamieniająca współrzędne ortokartezjańskie na współrzędne geodezyjne. Jako argumenty przyjmuje współrzędne X, Y, Z oraz parametry elipsoidy z wybranego modelu. Zwraca natomiast wartości fi, lambda oraz h.\newline
	phi lam XYZ - funkcja ta zamienia współrzędne geodezyjne na współrzędne ortokartezjańskie. Jako argumenty przyjmuje współrzędne fi, lambda, h oraz parametry elipsoidy z wybranego modelu. Zwraca natomiast współrzędne X, Y oraz Z.\newline
	transformacja neu - funkcja zmienia współrzędne ortokartezjańskie punktu na współrzędne horyzontalne względem podanego przez użytkownika środka układu. Jako argumenty przyjmuje współrzędne ortokartezjańskie punktu oraz współrzędne ortokartezjańskie środka układu. Zwraca natomiast współrzędne N, E, U.\newline
	GK2000 i GK92 to dwie podobnie działające funkcje. Jako argumenty przyjmują parametry elipsoidy oraz współrzędne geodezyjne punktu fi i lambda. Funkcje te najpierw zamieniają współrzędne geodezyjne na współrzędne na płaszczyźnie Gaussa-Krugera, a następnie wykorzystując odpowiednie matryce przeliczają te współrzędne odpowiednio na układy PL- 2000 i PL - 92.
	\item Następnie przechodzimy do zasadniczej części naszego programu "main". Na początku program wyświetla dostępne do wyboru modele i tworzony jest obiekt biorący dane z wybranego przez użytkownika modelu. Następnie wyświetlane są funkcje dostępne w programie, które użytkownik może wybrać. Dzięki zastosowaniu rzędu elif w dalszej części maina w zależności od wybranej przez użytkownika cyfry wykonywana jest odpowiednia transformacja. Nasz program pobiera nazwe pliku wejściowego oraz wyjściowego z wiersza poleceń dzięki zastosowaniu metody sys.argv. Uruchamiając program użytkownik musi podaj nazwę pliku z danymi oraz rządaną nazwę pliku wyjściowego.\newline
	Dzięki zastosowaniu pętli oraz funkcji sprawdzającej czy dany ciąg znaków jest liczbą float przy wczytywaniu danych plik wejściowy może mieć dowolną wielkość nagłówka, jednak dane muszą być rozdzielone przecinkami. (program zamienia każdą linijkę nie będącą danymi w listę z trzema zerami, które na późniejszym etapie nie są brane pod uwagę )
	\item Program zwraca błąd w przypadku podania niewłaściwej nazwy elipsoidy oraz w przypadku podania nieistniejącego numeru transformacji.Użyliśmy do teg metody raise. Wyświetlane jest również ostrzeżenie o nieprawidłowym działaniu transformacji do układów PL-2000 i PL-92 w przypadku wybrania elipsoidy Krassowskiego. Podczas używania programu należy uważać na podawanie do wczytania odpowiednich danych do odpowiednich transformacji, co opisaliśmy szerzej w pliku README.md dołączonym do projektu.
	\item Nasz program przetestowaliśmy wprowadzając różne pliki z danymi i przeprowadzając wszystkie dostępne w programie transformacje.
\end{itemize}
\subsection{Dokumentacja ćwiczenia}
Instrukcja do programu została zapisana w pliku Read.me w zdalnym repozytorium.
\section{Podsumowanie}
\subsection{Umiejętności nabyte podczas pracy nad projektem i obserwacje}
Wspólna, zdalna praca nad projektem nie byłaby możliwa bez programu GitHub, oraz zdalnego repozytorium. Umiejętność obsługi GitBash'a i komend pozwalających na "porozumiewanie się" dzięki temu programowi bardzo ułatwiły pracę i przyspieszyły postępy. Praca w zdalnym repozytorium pozwoliła trzymać wszystkie potrzebne pliki w jednym miejsu, dała możliwość śledzenia postępów i historii edycji.
\newline 
Użycie wiersza poleceń jako medium do obsługi programu pokazało nowe możliwości, a praca nad kodem w języku Python uststematyzowała wiedzę z zakresu programowania, rozszerzając ją o dodatkowe funkcje. Nowością było tworzenie funkcji uniwersalnej dla użytkownika, a nie dopasowanej pod konkretny plik, co wymagało innego podejścia do rozwiązania problemu. Przekazanie kodu w formacie .txt do wiersza poleceń wymagało prezycji i kontroli, gdyż w odróżnieniu od środowiska Spyder, pliki tekstowe nie wskazują błędów.
\subsection{Wnioski}
Zastosowanie użytych programów do stworzenia aplikacji ułatwiło pracę, oraz usystematyzowało wiedzę i nabyte dotychczas umiejętności, przy okazji pokazując ich praktyczne zastosowanie. Funkcjonalność Python'a, GitHub'a, oraz LaTeX'a pozwalają prościej i bardziej fachowo wykonać zadanie. Wymagania zawarte w instrukcji do ćwiczenia udowodniły, że zastosowanie odpowiednich funkcji, bibliotek i rozwiązań pozwala na rozwiązanie wielu problemów, daje dużą ilość możliwości, oraz widoczne efekty w postaci działającej aplikacji.
\subsection{Link do repozytorium}
Rezultat pracy nad programem został zapisany w zdalnym repozytorium, do którego link znajduje się poniżej:
\newline
\url{https://github.com/jakubCR7/Program1_infgeo2.git}%{https://github.com/jakubCR7/Program1_infgeo2.git}

\end{document}